\begin{enumerate}
\item (0.2) {\bf Determinant.}

Calculate the determinant of $\bA$.

\item (0.5) {\bf Matrix inverse.}

\begin{enumerate}
\item How do we know that the inverse of $\bA$ must exist?
\item Calculate $\bA^{-1}$.
\end{enumerate}

\item (1.0) {\bf Eigenvalues.}

Calculate the eigenvalues of $\bA$, assuming that one of the eigenvalues is either $-1$ or 1.

\item (1.8) {\bf Eigenvectors.}

\begin{enumerate}
\item (1.3) Calculate an eigenbasis $\bU$ of $\bA$. Solve by hand and show your work.
%
\begin{enumerate}
\item (0.7) $\bu_1$
\item (0.3) $\bu_2$
\item (0.3) $\bu_3$
\end{enumerate}

\item (0.2) Use Python to check that $\bA\bU = \bU\bD$, where $\bU$ is your eigenbasis and $\bD$ is your diagonal matrix with eigenvalues on the diagonal. (Note: I am {\bf not} asking to use Python command to obtain the eigenvalues.) Show your commands and output.

\item (0.2) Using Python, normalize the columns of $\bU$ and again check that $\bA\bU = \bU\bD$ with your new $\bU$. Show your commands and output.

\item (0.1) Describe (in words) the relationship between matrices $\bA$ and $\bD$ in terms of a linear transformation.

\end{enumerate}

%\pagebreak
\item (1.5) \ptag\ {\bf Orthogonalization.}

\begin{enumerate}
\item (1.2) Use the Gram-Schmidt procedure (by hand) to derive matrices $\bQ$ and $\bR$ of the decomposition of $\bA = \bQ\bR$. See \refFigii{fig:proj}{fig:ortho} as a guide.
\footnote{If you use \citet[][p.~326]{Aster}, note that his $\bw_k$ are {\em not} normalized. So his $\bw_k$ are my $\bh_k$, and my $\bw_k = \bh_k/\|\bh_k\|$. See \refFig{fig:ortho}.}

\item (0.1) Using Python or not, check that $\bQ$ is orthogonal.

\item (0.1) Using Python, check that $\bA = \bQ\bR$ with your $\bQ$ and $\bR$. Show your commands and output.

\item (0.1) Compare your results with those obtained using Python's factorization. Show your commands and output.
\end{enumerate}

\end{enumerate}
